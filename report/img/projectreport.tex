\documentclass[conference]{IEEEtran}
% The preceding line is only needed to identify funding in the first footnote. 
% If that is unneeded, please comment it out.
\usepackage{cite}
\usepackage{amsmath,amssymb,amsfonts}
\usepackage{algorithmic}
\usepackage{graphicx}
\usepackage{textcomp}
\usepackage{xcolor}
\usepackage{hyperref}
\usepackage{tabularx}
\begin{document}

\title{Applied AI in Biomedicine Project Report - Development of H.T.F. Deep-Learning Peak Classifier   }

\author{
Alberto Rota
    \IEEEauthorblockA{ \\
    \textit{Person Code: 10615751}\\
    \textit{Student Number: 964662} \\
    \href{mailto:alberto2.rota@mail.polimi.it}{alberto2.rota@mail.polimi.it}}
\and
Gabriele Santicchi 
    \IEEEauthorblockA{ \\
    \textit{Person Code: 10579046}\\
    \textit{Student Number: 969088}  \\
    \href{mailto:gabriele.santicchi@mail.polimi.it}{gabriele.santicchi@mail.polimi.it}}
}

\maketitle
\section{Context}
    The goal of this project is to build a classification model that properly 
    annotates each peaks as Normal, PAC (supraventricular beats) or PVC (ventricular beats).
    The available dataset is composed of 2-lead ECG signal and R peaks position of  
    105 patient recordings.  
\section{Introduction}
    Electrocardiogram (ECG) signals records the electrical activity of the human hearts 
    and consist of several waveforms (P, QRS, and T). The duration and shape of each 
    waveform and the distances between different peaks are used to diagnose cardiovascular heart
    diseases (CVD). Premature atrial contractions (PAC) and premature ventricular contractions (PVC) are among 
    the most common forms of arrhythmias; the first results from premature electrical activation 
    originating in the atria of the heart, while the second one is caused by a premature 
    electrical activation originating in the ventricles.
    As the presence of frequent PACs and PVCs are associated with a higher risk of unfavorable prognosis[1][2], 
    the accurate recognition of these abnormal beats is strictly required. 
    The aim of the this work is to design a deep-learning ensemble model that properly classifies
    all the peaks in a recording, without requiring the extraction of ECG features. 
    The History-Time-Frequency Classifier analyzes each samples in both the time and frequency domain, and take as input also the 
    labels assigned to the previous two samples. 
    The miss-classification error on the test set was around 3\%, highlighting the potential of our classifier. 


\subsection{Data Loading}
    sceggli il megglio
\subsection{Splitting}
    scegli eccampus
\subsection{Preprocessing}
    ti aspecto
\section{The H.T.F. Model}
1. Parlare del class imbalance? Undersampling? 
2. Mettere il graphviz del modello? 
\subsection{H as History}
During a first inspection phase, an analysis on the distribution of pathological beats has been performed. 
In fig. 1a, the histogram reports the abnormal interpeak distances computed over all the recordings, confirming that 
PACs and PVCs often occurs in repeated patterns[3][4]. For these reasons, the involvement of the labels assigned to 
the previous peak could help in predicting more precisely the current peak. Considering the interpeak distances distributions, 
in this classifier the two previous labels has been added as further inputs of the ensemble model.
\ref{fig:graphs}
\begin{figure}[]
    \includegraphics[width=\linewidth]{hist_plot.png}
    \caption{Pathological interpeak distances over the 105 recordings. rca troia non si legge niente }
    \label{fig:graphs}
\end{figure}
\subsection{T as Time}
\subsection{F as Frequency}
The last branch of this model consists in a frequency-domain classifier by a proper Fourier Transformation of the inputs.
In order to apply the Fourier Transform, the stationarity hypothesis has been assumed for each window. This is not an heavy assumption, since
the inputs length is very short and most part of the window consists in the peak recorded. 
The classifier consist of multiple CNN-ReLU-MaxPooling stacks. A low filter size has been chosen to get useful features in afew amount of samples at a time.
After the application of the FFT on the input signal, the output of the CNN layer is concatenated with the two history-labels as in the time model. 
Finally, a dense layer with a softmax AF provides the probability assigned to the three labels. 

\section{Training}
    
\section{Results}

% [1] Huang BT, Huang FY, Peng Y, et al. Relation of premature atrial complexes with stroke and death: 
% Systematic review and meta-analysis. Clin Cardiol. 2017;40(11):962-969. doi:10.1002/clc.22780
% [2] ideo Hirose, Shizukiyo Ishikawa, Tadao Gotoh, Tomoyuki Kabutoya, Kazunori Kayaba, Eiji Kajii,
% Cardiac mortality of premature ventricular complexes in healthy people in Japan,
% Journal of Cardiology, Volume 56, Issue 1,2010,Pages 23-26, ISSN 0914-5087,
% [3]Farzam K, Richards JR. Premature Ventricular Contraction. [Updated 2021 Dec 13]. 
% In: StatPearls [Internet]. Treasure Island (FL): StatPearls Publishing; 2022 Jan-. 
% Available from: https://www.ncbi.nlm.nih.gov/books/NBK532991/
% [4]Boon-Hor Chong, Vincent Pong, Kwok-Fai Lam, Shasha Liu, Ming-Liang Zuo, Yuk-Fai Lau, Chu-Pak Lau, 
% Hung-Fat Tse, Chung-Wah Siu, Frequent premature atrial complexes predict new occurrence of atrial fibrillation 
% and adverse cardiovascular events, EP Europace, Volume 14, Issue 7, July 2012, Pages 942–947, https://doi.org/10.1093/europace/eur389

\end{document}

